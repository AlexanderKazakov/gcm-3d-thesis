\begin{thebibliography}{99}
\addcontentsline{toc}{section}{Литература}

\bibitem{resler} Реслер И., Хардерс Х., Бекер М. Механическое поведение конструкционных материалов. Перевод с немецкого. - Долгопрудный, Издательский дом <<Интеллект>>, 2011, с. 91-115.

\bibitem{sedov} Седов Л. И. Механика сплошной среды. Том 1. — М. : Наука, 1970, с. 143.

\bibitem{kukudzhanov} Кукуджанов В.Н. Вычислительная механика сплошное сред. - М.: Издательство Физико-математической литературы, 2008, с. 32, с. 242, с. 271.

\bibitem{magomedov_kholodov_1969} Магомедов К.М., Холодов А.С. О построении разностных схем для уравнений гиперболического типа на основе характеристических соотношений, Ж. вычисл. матем. и матем. физ., 9:2 (1969), 373–386.

\bibitem{magomedov_kholodov_1988} Магомедов К.М., Холодов А.С. Сеточно-характеристические численные методы. — М.: Наука, 1988, 288 с.

\bibitem{petrov_kholodov} Петров И.Б., Холодов А.С. Численное исследование некоторых динамических задач механики деформируемого твёрдого тела сеточно-характеристическим методом, Ж. вычисл. матем. и матем. физ., 24:5 (1984), 722–739.

\bibitem{chelnokov_agapov} Агапов П.И., Челноков Ф.Б. Сравнительный анализ разностных схем для численного решения двумерных задач механики деформируемого твердого тела: Моделирование и обработка информации: М., МФТИ, 2003, 19 - 27.

\bibitem{chelnokov} Челноков Ф.Б., Явное представление сеточно-характеристических схем для уравнений упругости в двумерном и трехмерном пространствах, Матем. моделирование, 18:6 (2006), 96–108.

\bibitem{favorskaya} Петров И.Б., Фаворская А.В., Санников А.В., Квасов И.Е. Сеточно-характеристический метод с использованием интерполяции высоких порядков на тетраэдральных иерархических сетках с кратным шагом по времени, Матем. моделирование, 25:2 (2013), 42–52.

\bibitem{favorskaya_anysotropy} Фаворская А.В. Постановка задачи численного моделирования динамических процессов в сплошной линейно-упругой среде с анизотропией сеточно-характеристическим методом. // Труды 54-й научной конференции МФТИ: Проблемы фундаментальных и прикладных наук в современном информационном обществе. — 2011. — Т. 2. — с. 55 – 56.

\bibitem{favorskaya_disser} Фаворская А.В. Разработка численных методов для моделирования распространения упругих волн в неоднородных средах. Диссертация на соискание учёной степени кандидата физико-математических наук. — М. 2015.

\bibitem{kulikovskiy} Куликовский А.Г., Погорелов Н.В., Семёнов А.Ю. Математические вопросы численного решения гиперболических систем уравнений.

\bibitem{petrov_lobanov} Петров И.Б., Лобанов А.И. Лекции по вычислительной математике. – М. : Интернет-Ун-т информ. технологий, 2006.

\bibitem{prodaivoda} Александров К.С., Продайвода Г.Т. Анизотропия упругих свойств минералов и горных пород. – Новосибирск: Издательство СО РАН, 2000. 354 с.

\bibitem{akhmadeev} Ахмадеев Н.Х., Болотнова Р.Х. Распространение волн напряжений в слоистых средах при ударном нагружении (акустическое приближение) // ПМТФ. 1985. № 1. С. 125-133.

\bibitem{line_walker} Olivier Devillers, Sylvain Pion, Monique Teillaud. Walking in a triangulation. RR-4120, INRIA, 2001.

\bibitem{sibson} R. Sibson. A brief description of natural neighbour interpolation. In Vic Barnet, editor, Interpreting Multivariate Data, pages 21–36. John Wiley and Sons, Chichester, 1981.

\bibitem{cgal_interpolation} Julia Flötotto. 2D and Surface Function Interpolation. In CGAL User and Reference Manual. CGAL Editorial Board, 4.8 edition, 2016.

\bibitem{natural_neighbors} Tom Bobach, Georg Umlauf. Natural neighbor interpolation and order of continuity. In GI Lecture Notes in Informatics, Visualization of Large and Unstructured Data Sets. Springer: Heidelberg, 2006; 68–86.

\bibitem{cgal} CGAL, Computational Geometry Algorithms Library, http://www.cgal.org

\end{thebibliography}
