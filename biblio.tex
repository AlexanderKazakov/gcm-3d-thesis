\begin{thebibliography}{99}
\addcontentsline{toc}{section}{Литература}
\bibitem{kukudganov}Кукуджанов В.Н. Вычислительная механика сплошное сред. - М.: Издательство 
Физико-математической литературы, 2008, с. 32, с. 242, с. 271.
\bibitem{resler}И.Реслер, Х.Хардерс, М.Бекер Механическое поведение конструкционных материалов. Перевод с немецкого. - Долгопрудный, Издательский дом <<Интеллект>>, 2011, с. 91-115
\bibitem{novatsky}Новацкий В. К. Теория упругости. — М. : Мир, 1975, c. 105-107.
\bibitem{sedov}Седов Л. И. Механика сплошной среды. Том 1. — М. : Наука, 1970, с. 143.
\bibitem{rebotnov}Работнов Ю.Н. Механика деформируемого твёрдого тела. — М.: Наука, 1988. — 712 с.
\bibitem{belocerkovsky}Белоцерковский О.М. Численное моделирование в механике
сплошных сред. — М.: Физико-математическая литература. 1994, 442 с.
\bibitem{magomedov}Магомедов К.М., Холодов А.С. Сеточно-характеристические
численные методы. — М.: Наука, 1988, 288 с.
\bibitem{holodov}Холодов А.С., Холодов Я.А. О критериях монотонности разностных
схем для уравнений гиперболического типа. 
\bibitem{chelnokov}Челноков Ф.Б. Численное моделирование деформационных
процессов в средах со сложной структурой.
\bibitem{fedorenko}Федоренко Р.П. Введение в вычислительную физику. М.:
Изд-во Моск. физ. -техн. ин-та, 1994, 528 с.
\bibitem{chushkin}Чушкин П.И. Метод характеристик для пространственных сверхзвуковых течений. –  Труды ВЦ АН СССР, 1968, c. 121.
\bibitem{petrov_chelnokov}Петров И.Б., Челноков Ф.Б. Численное исследование волновых процессов и процессов разрушения в многослойных преградах // Журнал вычислительной математики и математической физики – 2003, том 43, N 10, с. 1562-1579.
\bibitem{matyushev_petrov}Matyushev N.G., Petrov I.B. Mathematical Simulation of Deformation and Wave Processes in Multilayered Structures // Computational Mathematics and Mathematical Physics – 2009, Vol. 49, N 9, P. 1615-1621.
\bibitem{petrov_tormasov_holodov}Петров  И.Б., Тормасов А.Г., Холодов А.С. О численном изучении нестационарных процессов в деформируемых средах многослойной структуры // Механика твердого тела – 1989, N 4, с. 89-95.
\bibitem{golubev_kvasov_petrov}Голубев В.И., Квасов И.Е., Петров И.Б. Воздействие природных катастроф на наземные сооружения // Математическое моделирование – 2011, том 23, N 8, с. 46-54.
\bibitem{agapov_belocerkovsky_petrov}Агапов П.И., Белоцерковский О.М., Петров И.Б. Численное моделирование последствий механического воздействия на мозг человека при черепно-мозговой травме // Журнал вычислительной математики и математической физики – 2006, том 46, N 9, с. 1711-1720.
\bibitem{petrov}Петров И.Б. Волновые и откольные явления в слоистых оболочках конечной толщины // Механика твердого тела – 1986, N 4, с. 118-124.
\end{thebibliography}
