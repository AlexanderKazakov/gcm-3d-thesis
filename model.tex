\section{Уравнения механики линейно-упругого тела}

\subsection{Общий вид уравнений}
\label{model}

Для математического моделирования волновых процессов в деформируемом твёрдом
теле используется система динамических уравнений \cite{sedov, kukudzhanov} в виде
\begin{eqnarray}
	\label{initial_equations}
	\rho\dot{v}_i=\nabla_j\sigma_{ij}+f_i & \textrm{(уравнения движения)}\nonumber\\
	\dot{\sigma}_{ij}=q_{ijkl}\dot{\varepsilon}_{kl}+F_{ij} & \textrm{(реологические соотношения).}
\end{eqnarray}

Здесь $\rho$ – плотность среды, $v_i$ – компоненты векторов скорости частиц среды,
$\sigma_{ij}$, $\varepsilon_{ij}$ -- компоненты симметричных тензоров напряжений и деформаций,
$\nabla_j$ – производная по $j$-й координате, $f_i$ – массовые
силы, действующие на единицу объёма, $F_{ij}$ -- силы, обусловленные вязкостью, $q_{ijkl}$ -- 
тензор упругих постоянных.

В случае малых деформаций тензор скоростей деформаций $e_{ij}=\dot{\varepsilon}_{ij}$ 
выражается через компоненты скорости смещения линейным образом:
\begin{equation}
	e_{ij}=\frac{1}{2}(\nabla_j v_i+\nabla_i v_j).
\end{equation}

Компоненты тензора 4-го порядка $q_{ijkl}$, сил вязкости $F_{ij}$ и плотности $\rho$ определяются реологией среды. 

\subsection{Случай изотропного линейно-упругого тела}
Для невязкого изотропного линейно-упругого материала
\begin{eqnarray}
	\label{isotropic_tensor}
	q_{ijkl}=\lambda\delta_{ij}\delta_{kl}+\mu(\delta_{ik}\delta_{jl}+\delta_{il}
	\delta_{jk}) & \textrm {(изотропия)} \nonumber\\
	F_{ij}=0 & \textrm {(отсутствует вязкость)} \\
	\rho=const & \textrm {(изменения пренебрежимо малы)}.
\end{eqnarray}

В этом соотношении, которое обобщает закон Гука, $\lambda$ и $\mu$ -- параметры
Ляме, связанные с более известными модулем Юнга $E$ и коэффициентом Пуассона $\nu$ (модулем сдвига $G$) следующими соотношениями:
\begin{eqnarray}
\label{lame_parameters}
	\lambda &= \frac{E\nu}{(1+\nu)(1-2\nu)}
	\nonumber\\
	\mu &= G=\frac{E}{2(1+\nu)}.
\end{eqnarray}


В этом случае уравнения \eqref{initial_equations} принимают вид:
\begin{align}
	\label{simple_equations}
	\frac{\partial{v_x}}{\partial{t}}&=\frac{1}{\rho}(\frac{\partial{\sigma_{xx}}}{\partial{x}}+\frac{\partial{\sigma_{xy}}}{\partial{y}}+\frac{\partial{\sigma_{xz}}}{\partial{z}})
	\nonumber\\
	\frac{\partial{v_y}}{\partial{t}}&=\frac{1}{\rho}(\frac{\partial{\sigma_{xy}}}{\partial{x}}+\frac{\partial{\sigma_{yy}}}{\partial{y}}+\frac{\partial{\sigma_{yz}}}{\partial{z}})
	\nonumber\\
	\frac{\partial{v_z}}{\partial{t}}&=\frac{1}{\rho}(\frac{\partial{\sigma_{xz}}}{\partial{x}}+\frac{\partial{\sigma_{yz}}}{\partial{y}}+\frac{\partial{\sigma_{zz}}}{\partial{z}})
	\nonumber\\
	\frac{\partial{\sigma_{xx}}}{\partial{t}}&=(\lambda+2\mu)\frac{\partial{v_x}}{\partial{x}}+\lambda\frac{\partial{v_y}}{\partial{y}}+\lambda\frac{\partial{v_z}}{\partial{z}}
	\nonumber\\
	\frac{\partial{\sigma_{xy}}}{\partial{t}}&=\mu(\frac{\partial{v_x}}{\partial{y}}+\frac{\partial{v_y}}{\partial{x}})
	\nonumber\\
	\frac{\partial{\sigma_{xz}}}{\partial{t}}&=\mu(\frac{\partial{v_x}}{\partial{z}}+\frac{\partial{v_z}}{\partial{x}})
	\nonumber\\
	\frac{\partial{\sigma_{yy}}}{\partial{t}}&=\lambda\frac{\partial{v_x}}{\partial{x}}+(\lambda+2\mu)\frac{\partial{v_y}}{\partial{y}}+\lambda\frac{\partial{v_z}}{\partial{z}}
	\nonumber\\
	\frac{\partial{\sigma_{yz}}}{\partial{t}}&=\mu(\frac{\partial{v_z}}{\partial{y}}+\frac{\partial{v_y}}{\partial{z}})
	\nonumber\\
	\frac{\partial{\sigma_{zz}}}{\partial{t}}&=\lambda\frac{\partial{v_x}}{\partial{x}}+\lambda\frac{\partial{v_y}}{\partial{y}}+(\lambda+2\mu)\frac{\partial{v_z}}{\partial{z}}
\end{align}

Обозначив искомый вектор $\vec{u}=\{v_x,v_y,v_z,\sigma_{xx},\sigma_{xy},\sigma_{xz},\sigma_{yy},\sigma_{yz},\sigma_{zz}\}^T$, уравнения \eqref{initial_equations} можно переписать в матричной форме \cite{petrov_kholodov}:

\begin{equation}
	\label{simple_matrix_equation}
	\frac{\partial\vec{u}}{\partial{t}}+\mathbf{A}_x\frac{\partial\vec{u}}{\partial{x}}+
	\mathbf{A}_y\frac{\partial\vec{u}}{\partial{y}}+
	\mathbf{A}_z\frac{\partial\vec{u}}{\partial{z}}=0,
\end{equation}
где матрицы $\mathbf{A}_x$, $\mathbf{A}_y$, $\mathbf{A}_z$ принимают следующий вид:

\begin{align}
\label{isotropic_mat1}
\mathbf{A}_x =
\left( \begin{array}{cccccccccccc}
0 & 0 & 0 & -\frac 1 \rho & 0 & 0 & 0 & 0 & 0 \\ 
0 & 0 & 0 & 0 & -\frac 1 \rho & 0 & 0 & 0 & 0 \\ 
0 & 0 & 0 & 0 & 0 & -\frac 1 \rho & 0 & 0 & 0 \\ 
-(\lambda+2\mu) & 0 & 0 & 0 & 0 & 0 & 0 & 0 & 0 \\ 
0 & -\mu & 0 & 0 & 0 & 0 & 0 & 0 & 0 \\ 
0 & 0 & -\mu & 0 & 0 & 0 & 0 & 0 & 0 \\ 
-\lambda & 0 & 0 & 0 & 0 & 0 & 0 & 0 & 0 \\ 
0 & 0 & 0 & 0 & 0 & 0 & 0 & 0 & 0 \\ 
-\lambda & 0 & 0 & 0 & 0 & 0 & 0 & 0 & 0  
\end{array} \right),
\end{align} 
\begin{align}
\label{isotropic_mat2}
\mathbf{A}_y =
\left( \begin{array}{cccccccccccc}
0 & 0 & 0 & 0 & -\frac 1 \rho & 0 & 0 & 0 & 0 \\ 
0 & 0 & 0 & 0 & 0 & 0 & -\frac 1 \rho & 0 & 0 \\ 
0 & 0 & 0 & 0 & 0 & 0 & 0 & -\frac 1 \rho & 0 \\ 
0 & -\lambda & 0 & 0 & 0 & 0 & 0 & 0 & 0 \\ 
-\mu & 0 & 0 & 0 & 0 & 0 & 0 & 0 & 0 \\ 
0 & 0 & 0 & 0 & 0 & 0 & 0 & 0 & 0 \\ 
0 & -(\lambda+2\mu) & 0 & 0 & 0 & 0 & 0 & 0 & 0 \\ 
0 & 0 & -\mu & 0 & 0 & 0 & 0 & 0 & 0 \\ 
0 & -\lambda & 0 & 0 & 0 & 0 & 0 & 0 & 0  
\end{array} \right),
\end{align}
\begin{align}
\label{isotropic_mat3}
\mathbf{A}_z =
\left( \begin{array}{cccccccccccc}
0 & 0 & 0 & 0 & 0 & -\frac 1 \rho & 0 & 0 & 0 \\ 
0 & 0 & 0 & 0 & 0 & 0 & 0 & -\frac 1 \rho & 0 \\ 
0 & 0 & 0 & 0 & 0 & 0 & 0 & 0 & -\frac 1 \rho \\ 
0 & 0 & -\lambda & 0 & 0 & 0 & 0 & 0 & 0 \\ 
0 & 0 & 0 & 0 & 0 & 0 & 0 & 0 & 0 \\ 
-\mu & 0 & 0 & 0 & 0 & 0 & 0 & 0 & 0 \\ 
0 & 0 & -\lambda & 0 & 0 & 0 & 0 & 0 & 0 \\ 
0 & -\mu & 0 & 0 & 0 & 0 & 0 & 0 & 0 \\ 
0 & 0 & -(\lambda+2\mu) & 0 & 0 & 0 & 0 & 0 & 0  
\end{array} \right).
\end{align}\\


\subsection{Случай произвольно анизотропного линейно-упругого тела}
Для материала с произвольным типом анизотропии тензор упругих постоянных $q_{ijkl}$ обладает 21 независимым параметром.
Это видно из следующих рассуждений \cite{prodaivoda}. Вообще говоря, число его компонент $3^4 = 81$. Однако симметричные в рассматриваемой модели тензоры напряжений и деформаций имеют не 9, а 6 независимых компонент. Остаётся 36 независимых компонент $q_{ijkl}$. Теперь рассмотрим выражение для потенциала упругой энергии $W$:

\begin{eqnarray}
dW &= \sigma_{ij} d\varepsilon_{ij}, \\
q_{ijkl} &= \frac{\partial{\sigma_{ij}}}{\partial{\varepsilon_{kl}}}, \\
q_{ijkl} &= \frac{\partial^{2}{W}}{\partial{\varepsilon_{ij}\varepsilon_{kl}}}.
\end{eqnarray}

Из независимости второй производной от порядка дифференцирования следует $q_{ijkl} = q_{klij}$. Теперь зависимость $\sigma$ от $\varepsilon$ можно записать в более компактном виде:
\begin{align}
\left( \begin{array}{cccccccccccc}
\sigma_{11} \\
\sigma_{22} \\
\sigma_{33} \\
\sigma_{23} \\
\sigma_{13} \\
\sigma_{12} 
\end{array} \right){}
= \left( \begin{array}{cccccccccccc}
c_{11} & c_{12} & c_{13} & c_{14} & c_{15} & c_{16} \\ 
c_{12} & c_{22} & c_{23} & c_{24} & c_{25} & c_{26} \\ 
c_{13} & c_{23} & c_{33} & c_{34} & c_{35} & c_{36} \\ 
c_{14} & c_{24} & c_{34} & c_{44} & c_{45} & c_{46} \\ 
c_{15} & c_{25} & c_{35} & c_{45} & c_{55} & c_{56} \\ 
c_{16} & c_{26} & c_{36} & c_{46} & c_{56} & c_{66}
\end{array} \right){}
\left( \begin{array}{cccccccccccc}
\varepsilon_{11} \\
\varepsilon_{22} \\
\varepsilon_{33} \\
\varepsilon_{23} \\
\varepsilon_{13} \\
\varepsilon_{12}
\end{array} \right)
\end{align}


Такми образом, для случая произвольной анизотропии уравнения \eqref{initial_equations} принимают вид:
\begin{small}
\begin{align}
	\label{anisotropic_equations}
	\frac{\partial{v_x}}{\partial{t}}&=\frac{1}{\rho}(\frac{\partial{\sigma_{xx}}}{\partial{x}}+\frac{\partial{\sigma_{xy}}}{\partial{y}}+\frac{\partial{\sigma_{xz}}}{\partial{z}})
	\nonumber\\
	\frac{\partial{v_y}}{\partial{t}}&=\frac{1}{\rho}(\frac{\partial{\sigma_{xy}}}{\partial{x}}+\frac{\partial{\sigma_{yy}}}{\partial{y}}+\frac{\partial{\sigma_{yz}}}{\partial{z}})
	\nonumber\\
	\frac{\partial{v_z}}{\partial{t}}&=\frac{1}{\rho}(\frac{\partial{\sigma_{xz}}}{\partial{x}}+\frac{\partial{\sigma_{yz}}}{\partial{y}}+\frac{\partial{\sigma_{zz}}}{\partial{z}})
	\nonumber\\
	\frac{\partial{\sigma_{xx}}}{\partial{t}}&=c_{11}\frac{\partial{v_x}}{\partial{x}}+c_{12}\frac{\partial{v_y}}{\partial{y}}+c_{13}\frac{\partial{v_z}}{\partial{z}}+c_{14}(\frac{\partial{v_z}}{\partial{y}}+\frac{\partial{v_y}}{\partial{z}})+c_{15}(\frac{\partial{v_z}}{\partial{x}}+\frac{\partial{v_x}}{\partial{z}})+c_{16}(\frac{\partial{v_y}}{\partial{x}}+\frac{\partial{v_x}}{\partial{y}})
	\nonumber\\
	\frac{\partial{\sigma_{yy}}}{\partial{t}}&=c_{12}\frac{\partial{v_x}}{\partial{x}}+c_{22}\frac{\partial{v_y}}{\partial{y}}+c_{23}\frac{\partial{v_z}}{\partial{z}}+c_{24}(\frac{\partial{v_z}}{\partial{y}}+\frac{\partial{v_y}}{\partial{z}})+c_{25}(\frac{\partial{v_z}}{\partial{x}}+\frac{\partial{v_x}}{\partial{z}})+c_{26}(\frac{\partial{v_y}}{\partial{x}}+\frac{\partial{v_x}}{\partial{y}})
	\nonumber\\
	\frac{\partial{\sigma_{zz}}}{\partial{t}}&=c_{13}\frac{\partial{v_x}}{\partial{x}}+c_{23}\frac{\partial{v_y}}{\partial{y}}+c_{33}\frac{\partial{v_z}}{\partial{z}}+c_{34}(\frac{\partial{v_z}}{\partial{y}}+\frac{\partial{v_y}}{\partial{z}})+c_{35}(\frac{\partial{v_z}}{\partial{x}}+\frac{\partial{v_x}}{\partial{z}})+c_{36}(\frac{\partial{v_y}}{\partial{x}}+\frac{\partial{v_x}}{\partial{y}})
	\nonumber\\
	\frac{\partial{\sigma_{yz}}}{\partial{t}}&=c_{14}\frac{\partial{v_x}}{\partial{x}}+c_{24}\frac{\partial{v_y}}{\partial{y}}+c_{34}\frac{\partial{v_z}}{\partial{z}}+c_{44}(\frac{\partial{v_z}}{\partial{y}}+\frac{\partial{v_y}}{\partial{z}})+c_{45}(\frac{\partial{v_z}}{\partial{x}}+\frac{\partial{v_x}}{\partial{z}})+c_{46}(\frac{\partial{v_y}}{\partial{x}}+\frac{\partial{v_x}}{\partial{y}})
	\nonumber\\
	\frac{\partial{\sigma_{xz}}}{\partial{t}}&=c_{15}\frac{\partial{v_x}}{\partial{x}}+c_{25}\frac{\partial{v_y}}{\partial{y}}+c_{35}\frac{\partial{v_z}}{\partial{z}}+c_{45}(\frac{\partial{v_z}}{\partial{y}}+\frac{\partial{v_y}}{\partial{z}})+c_{55}(\frac{\partial{v_z}}{\partial{x}}+\frac{\partial{v_x}}{\partial{z}})+c_{56}(\frac{\partial{v_y}}{\partial{x}}+\frac{\partial{v_x}}{\partial{y}})
	\nonumber\\
	\frac{\partial{\sigma_{xy}}}{\partial{t}}&=c_{16}\frac{\partial{v_x}}{\partial{x}}+c_{26}\frac{\partial{v_y}}{\partial{y}}+c_{36}\frac{\partial{v_z}}{\partial{z}}+c_{46}(\frac{\partial{v_z}}{\partial{y}}+\frac{\partial{v_y}}{\partial{z}})+c_{56}(\frac{\partial{v_z}}{\partial{x}}+\frac{\partial{v_x}}{\partial{z}})+c_{66}(\frac{\partial{v_y}}{\partial{x}}+\frac{\partial{v_x}}{\partial{y}})
\end{align}
\end{small}
	
Записывая эти уравнения в матричном виде \eqref{simple_matrix_equation}, получаем следующие выражения для матриц:	
\begin{align}
\label{anisotropic_mat1}	
\mathbf{A}_x = - 
\left( \begin{array}{cccccccccccc}
0 & 0 & 0 & \frac 1 \rho & 0 & 0 & 0 & 0 & 0 \\ 
0 & 0 & 0 & 0 & \frac 1 \rho & 0 & 0 & 0 & 0 \\ 
0 & 0 & 0 & 0 & 0 & \frac 1 \rho & 0 & 0 & 0 \\ 
c_{11} & c_{16} & c_{15} & 0 & 0 & 0 & 0 & 0 & 0 \\ 
c_{16} & c_{66} & c_{56} & 0 & 0 & 0 & 0 & 0 & 0 \\
c_{15} & c_{56} & c_{55} & 0 & 0 & 0 & 0 & 0 & 0 \\ 
c_{12} & c_{26} & c_{25} & 0 & 0 & 0 & 0 & 0 & 0 \\ 
c_{14} & c_{46} & c_{45} & 0 & 0 & 0 & 0 & 0 & 0 \\ 
c_{13} & c_{36} & c_{35} & 0 & 0 & 0 & 0 & 0 & 0
\end{array} \right),
\end{align} 
\begin{align}
\label{anisotropic_mat2}
\mathbf{A}_y = - 
\left( \begin{array}{cccccccccccc}
0 & 0 & 0 & 0 & \frac 1 \rho & 0 & 0 & 0 & 0 \\ 
0 & 0 & 0 & 0 & 0 & 0 & \frac 1 \rho & 0 & 0 \\ 
0 & 0 & 0 & 0 & 0 & 0 & 0 & \frac 1 \rho & 0 \\ 
c_{16} & c_{12} & c_{14} & 0 & 0 & 0 & 0 & 0 & 0 \\ 
c_{66} & c_{26} & c_{46} & 0 & 0 & 0 & 0 & 0 & 0 \\
c_{56} & c_{25} & c_{45} & 0 & 0 & 0 & 0 & 0 & 0 \\
c_{26} & c_{22} & c_{24} & 0 & 0 & 0 & 0 & 0 & 0 \\ 
c_{46} & c_{24} & c_{44} & 0 & 0 & 0 & 0 & 0 & 0 \\
c_{36} & c_{23} & c_{34} & 0 & 0 & 0 & 0 & 0 & 0   
\end{array} \right),
\end{align}
\begin{align}
\label{anisotropic_mat3}
\mathbf{A}_z = - 
\left( \begin{array}{cccccccccccc}
0 & 0 & 0 & 0 & 0 & \frac 1 \rho & 0 & 0 & 0 \\ 
0 & 0 & 0 & 0 & 0 & 0 & 0 & \frac 1 \rho & 0 \\ 
0 & 0 & 0 & 0 & 0 & 0 & 0 & 0 & \frac 1 \rho \\ 
c_{15} & c_{14} & c_{13} & 0 & 0 & 0 & 0 & 0 & 0 \\ 
c_{56} & c_{46} & c_{36} & 0 & 0 & 0 & 0 & 0 & 0 \\
c_{55} & c_{45} & c_{35} & 0 & 0 & 0 & 0 & 0 & 0 \\ 
c_{25} & c_{24} & c_{23} & 0 & 0 & 0 & 0 & 0 & 0 \\ 
c_{45} & c_{44} & c_{34} & 0 & 0 & 0 & 0 & 0 & 0 \\ 
c_{35} & c_{34} & c_{33} & 0 & 0 & 0 & 0 & 0 & 0  
\end{array} \right).
\end{align}\\


\subsection{Отдельные виды анизотропии}
\subsubsection{Изотропный случай как подвид анизотропного}
Из условия изотропии -- неизменности свойств материала при произвольных поворотах -- следует, что число независимых компонент тензора $q_{ijkl}$ сокращается до двух:
\begin{align}
\left( \begin{array}{cccccccccccc}
c_{11} & c_{12} & c_{12} & 0 & 0 & 0 \\ 
c_{12} & c_{11} & c_{12} & 0 & 0 & 0 \\ 
c_{12} & c_{12} & c_{11} & 0 & 0 & 0 \\ 
0 & 0 & 0 & c_{44} & 0 & 0 \\ 
0 & 0 & 0 & 0 & c_{44} & 0 \\ 
0 & 0 & 0 & 0 & 0 & c_{44}
\end{array} \right){},
\end{align}
где $ c_{44} = \frac{c_{11} - c_{12}}{2} $, причём $c_{12} = \lambda$ и $c_{44} = \mu$ -- параметры Ламе.

\subsubsection{Орторомбическая анизотропия}
Орторомбическая анизотропия -- различные свойства материала вдоль трёх взаимно перпендикулярных направлений. Число независимых компонент сокращается до девяти.
\begin{align}
\label{orthorombic_tensor}
\left( \begin{array}{cccccccccccc}
c_{11} & c_{12} & c_{13} & 0 & 0 & 0 \\ 
c_{12} & c_{22} & c_{23} & 0 & 0 & 0 \\ 
c_{13} & c_{23} & c_{33} & 0 & 0 & 0 \\ 
0 & 0 & 0 & c_{44} & 0 & 0 \\ 
0 & 0 & 0 & 0 & c_{55} & 0 \\ 
0 & 0 & 0 & 0 & 0 & c_{66}
\end{array} \right){}
\end{align}

\subsubsection{Трансверсально-изотропное тело}
Трансверсальная анизотропия -- отличные свойства вдоль одной выделенной оси. Это частный случай орторомбической. Число независимых компонент -- пять.	
\begin{align}
\label{vert_trans_tensor}
\left( \begin{array}{cccccccccccc}
c_{11} & c_{12} & c_{13} & 0 & 0 & 0 \\ 
c_{12} & c_{11} & c_{13} & 0 & 0 & 0 \\ 
c_{13} & c_{13} & c_{33} & 0 & 0 & 0 \\ 
0 & 0 & 0 & c_{44} & 0 & 0 \\ 
0 & 0 & 0 & 0 & c_{44} & 0 \\ 
0 & 0 & 0 & 0 & 0 & c_{66}
\end{array} \right){},
\end{align}
где $ c_{66} = \frac{c_{11} - c_{12}}{2} $.


\subsection{Спектральное разложение матриц из уравнения}
Как будет показано ниже, для реализации сеточно\hyp{}характеристического метода необходимо диагонализовать матрицы из \eqref{simple_matrix_equation},
то есть представить их в виде $\mathbf{A} = \mathbf{\Omega}^{-1} \mathbf{L} \mathbf{\Omega},$  где $\mathbf{L}$ -- диагональная матрица собственных значений матрицы $\mathbf{A}$, $\mathbf{\Omega}^{-1}$ -- матрица собственных векторов матрицы $\mathbf{A}$, $\mathbf{\Omega}$ -- матрица собственных строк матрицы $\mathbf{A}$.

В случае орторомбической анизотропии материала и совпадения координатных осей с главными направлениями анизотропии разложение выписывается аналитически и представлено в \ref{application1}. Трансверсально-изотропный и изотропный сводятся к орторомбическому как его частные случаи.

Однако общий случай, когда имеется 21 независимая компонента или главные оси материала повёрнуты по отношению к координатным, требует больших усилий. Итак, проведём диагонализацию матриц \eqref{anisotropic_mat1}, \eqref{anisotropic_mat2}, \eqref{anisotropic_mat3}, как было предложено в \cite{favorskaya_anysotropy}.
Пусть  $\lambda^{2} = t$, где $\lambda$ -- собственное значение матрицы $\mathbf{A}_x$, тогда из векового уравнения имеем:
\begin{small}
\begin{align}	
	\label{eigenvalue_equation1}
	t^{3} &- \frac{1}{\rho}(c_{11} + c_{55} + c_{66})\;t^{2} - \frac{1}{\rho^{2}}(c_{15}^{2} - c_{11}c_{55} + c_{16}^{2} - c_{11}c_{66} + c_{56}^{2} - c_{55}c_{66})\;t\;+ \nonumber\\
	&+ \frac{1}{\rho^{3}}((c_{56}^{2} - c_{55}c_{66})c_{11} + (c_{16}c_{55} - c_{15}c_{56})c_{16} + (c_{15}c_{66} - c_{16}c_{56})c_{15}) = 0.
\end{align}
\end{small}
Решение может быть получено, например, с помощью тригонометрической формулы Виета или других методов. После чего собственные значения $\mathbf{A}_x$:
\begin{align}
	\left\{\sqrt{t_{11}},\;-\sqrt{t_{11}},\;\sqrt{t_{12}},\;-\sqrt{t_{12}},\;\sqrt{t_{13}},\;-\sqrt{t_{13}},\;0,\;0,\;0\right\},
\end{align}
где $t_{11}$, $t_{12}$, $t_{13}$ -- действительные положительные корни \eqref{eigenvalue_equation1}. Требования к действительности и положительности корней \eqref{eigenvalue_equation1} являются фактически требованием гиперболичности системы уравнений.
	
Аналогично для $\mathbf{A}_y$:
\begin{small}
\begin{align}	
	\label{eigenvalue_equation2}
	t^{3} &- \frac{1}{\rho}(c_{22} + c_{44} + c_{66})\;t^{2} - \frac{1}{\rho^{2}}(c_{24}^{2} - c_{22}c_{44} + c_{26}^{2} - c_{22}c_{66} + c_{46}^{2} - c_{44}c_{66})\;t\;+ \nonumber\\
	&+ \frac{1}{\rho^{3}}((c_{46}^{2} - c_{44}c_{66})c_{22} + (c_{26}c_{44} - c_{24}c_{46})c_{26} + (c_{24}c_{66} - c_{26}c_{46})c_{24}) = 0.
\end{align}
\end{small}
\begin{align}
	\left\{\sqrt{t_{21}},\;-\sqrt{t_{21}},\;\sqrt{t_{22}},\;-\sqrt{t_{22}},\;\sqrt{t_{23}},\;-\sqrt{t_{23}},\;0,\;0,\;0\right\},
\end{align}
Аналогично для $\mathbf{A}_z$:
\begin{small}
\begin{align}	
	\label{eigenvalue_equation3}
	t^{3} &- \frac{1}{\rho}(c_{33} + c_{44} + c_{55})\;t^{2} - \frac{1}{\rho^{2}}(c_{34}^{2} - c_{33}c_{44} + c_{35}^{2} - c_{33}c_{55} + c_{45}^{2} - c_{44}c_{55})\;t\;+ \nonumber\\
	&+ \frac{1}{\rho^{3}}((c_{45}^{2} - c_{44}c_{55})c_{33} + (c_{35}c_{44} - c_{34}c_{45})c_{35} + (c_{34}c_{55} - c_{35}c_{45})c_{34}) = 0.
\end{align}
\end{small}
\begin{align}
	\left\{\sqrt{t_{31}},\;-\sqrt{t_{31}},\;\sqrt{t_{32}},\;-\sqrt{t_{32}},\;\sqrt{t_{33}},\;-\sqrt{t_{33}},\;0,\;0,\;0\right\},
\end{align}
	
Нахождение собственных векторов \eqref{anisotropic_mat1}-\eqref{anisotropic_mat3} благодаря разреженности матриц сводится от СЛАУ $9\times9$ к СЛАУ $3\times3$.
Например, для матрицы $\mathbf{A}_x$:
\begin{align}
\label{eigenvector_equation}
\left( \begin{array}{cccccccccccc}
\lambda & 0 & 0 & \frac 1 \rho & 0 & 0 & 0 & 0 & 0 \\ 
0 & \lambda & 0 & 0 & \frac 1 \rho & 0 & 0 & 0 & 0 \\ 
0 & 0 & \lambda & 0 & 0 & \frac 1 \rho & 0 & 0 & 0 \\ 
c_{11} & c_{16} & c_{15} & \lambda & 0 & 0 & 0 & 0 & 0 \\ 
c_{16} & c_{66} & c_{56} & 0 & \lambda & 0 & 0 & 0 & 0 \\
c_{15} & c_{56} & c_{55} & 0 & 0 & \lambda & 0 & 0 & 0 \\ 
c_{12} & c_{26} & c_{25} & 0 & 0 & 0 & \lambda & 0 & 0 \\ 
c_{14} & c_{46} & c_{45} & 0 & 0 & 0 & 0 & \lambda & 0 \\ 
c_{13} & c_{36} & c_{35} & 0 & 0 & 0 & 0 & 0 & \lambda
\end{array} \right){}
\left( \begin{array}{cccccccccccc}
l_1 \\
l_2 \\
l_3 \\
l_4 \\
l_5 \\
l_6 \\
l_7 \\
l_8 \\
l_9
\end{array} \right){}
 = 0,
\end{align}
где $\vec{l}$ -- собственный вектор $\mathbf{A}_x$, соответствующий собственному значению $\lambda$.
Выражая $l_4$, $l_5$, $l_6$ из первых трёх строк и подставляя их в 4-ую, 5-ую и 6-ую строки, получим СЛАУ на компоненты $l_1$, $l_2$, $l_3$:
\begin{align}
\label{simple_eigenvector_equation}
\left( \begin{array}{cccccccccccc}
c_{11} + \rho\lambda^{2} & c_{16} & c_{15} \\ 
c_{16} & c_{66} + \rho\lambda^{2} & c_{56} \\ 
c_{15} & c_{56} & c_{55} + \rho\lambda^{2} 
\end{array} \right){}
\left( \begin{array}{cccccccccccc}
l_1 \\
l_2 \\
l_3
\end{array} \right){}
 = 0
\end{align}
Полученная матрица \eqref{simple_eigenvector_equation} имеет ранг либо 1, что соответствует корню кратности 2, либо ранг 2, что соответствует корню кратности 1.
Находя невырожденный минор соответствующей размерности и решая подсистему, получаем $l_1$, $l_2$, $l_3$, а затем и весь вектор $\vec{l}$.


\subsection{Преобразование тензора упругих постоянных при повороте}
Запишем изменение тензора упругих постоянных при повороте материала относительно системы координат.

Пусть $\theta_{x}$, $\theta_{y}$, $\theta_{z}$ -- углы поворота материала вокруг соответсвующих осей.
Имеем матрицы поворотов $\mathbf{G}_x$, $\mathbf{G}_y$, $\mathbf{G}_z$. Например, $\mathbf{G}_x$:
\begin{align}
\mathbf{G}_x =
\left( \begin{array}{cccccccccccc}
1 & 0 & 0 \\ 
0 & \cos \theta_{x} & -\sin \theta_{x} \\ 
0 & \sin \theta_{x} & \cos \theta_{x}
\end{array} \right),
\end{align}

Итоговая матрица преобразования базиса $\mathbf{G} = \mathbf{G}_{x}\mathbf{G}_{y}\mathbf{G}_{z}$, если поворот производился сначала вокруг $z$, потом вокруг $y$, потом вокруг $x$.

Выражение для тензора упругих постоянных $q_{ijkl}$ при таком повороте определяется правилом преобразования тензоров:
\begin{align}
	q_{mnpq} = \sum_{i,\;j,\;k,\;l = 1}^{3} G_{mi}\;G_{nj}\;G_{pk}\;G_{ql}\;q_{ijkl}.
\end{align}








