\section*{Заключение}
\addcontentsline{toc}{section}{Заключение}
\setcounter{subsection}{0}

	Можно выделить основные результаты данной работы:	
\begin{enumerate}
	\item На начальном этапе было проведено изучение литературы по механике сплошных сред, численным методам расчёта упругих деформаций в твёрдых изотропных и анизотроных материалах, подробное изучение сеточно-характерестического метода.
			Далее были изучены особенности механики ПКМ и численного моделирования слоистых структур. Был подробно рассмотрен ряд научных статей по моделированию динамических задач МДТТ.
	\item Написана модификация существующего пакета расчёта динамических задач МДТТ сеточно-характеристическим методом позволяющая производить расчёты анизотропных тел.
	\item Поставлен и успешно проведён ряд тестовых задач распространения P и S волн в анизотропном теле для проверки написанной модификации.
	\item Произведены расчёты точечного взрыва в однородном изотропном и анизотропном телах, демонстрирующие наличие дополнительных сдвиговых волн в анизотропном случае (по сравнению с изотропным). Эти волны были ранее предсказаны в \cite{ogurtsov}.
	\item Выполнен ряд расчётов динамического нагружения двухслойной конструкции, иллюстрирующих изменение формы фронтов волн при переходе между слоями.
	\item Получен эффект расслоения контакта двух анизотропных слоёв при динамическом нагружении. Форма разрушенной области совпадает результатами других численных и экспериментальных рассмотрений \cite{serge}.
\end{enumerate}

	Ряд основных результов моделирования волн упругих деформаций в анизотропных телах был представлен в работе:
	\begin{itemize}
		\item Петров И.Б., Фаворская А.В., Васюков А.В., Ермаков А.С., Беклемышева К.А., Казаков А.О., Новиков А.В. <<О численном моделировании волновых процессов в анизотропных средах>>. // Журнал <<Доклады Академии Наук>>. - 2014.
	\end{itemize}
