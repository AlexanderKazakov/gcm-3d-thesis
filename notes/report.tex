\documentclass[a4paper,14pt]{extarticle}
%\documentclass[a4paper]{article}
\usepackage[T2A]{fontenc}
\usepackage{multirow}
\usepackage[utf8]{inputenc}
\usepackage[english,russian]{babel}
\usepackage{indentfirst}
\usepackage{amssymb}
\usepackage{amsfonts}
\usepackage{amsmath}
\usepackage{mathtext}
\usepackage{cite}
\usepackage{enumerate}
\usepackage{hyphenat}
\usepackage{float}
\usepackage[top=1.5cm,bottom=1.5cm,left=2.5cm,right=1.5cm]{geometry}
\usepackage[unicode]{hyperref}
\usepackage{graphicx}
\usepackage{color}
\usepackage[colorinlistoftodos]{todonotes}
\usepackage[format=hang, labelsep=period, margin=10pt, figurename=Рис.]{caption}
\usepackage{listings}
\usepackage{subcaption}
%\usepackage{showkeys}
\usepackage[onehalfspacing]{setspace}

\definecolor{dkgreen}{rgb}{0,0.6,0}
\definecolor{gray}{rgb}{0.5,0.5,0.5}
\definecolor{mauve}{rgb}{0.58,0,0.82}
 
\lstset{ %
  language=C++,                % the language of the code
  basicstyle=\footnotesize,           % the size of the fonts that are used for the code
%  numbers=left,                   % where to put the line-numbers
%  numberstyle=\tiny\color{gray},  % the style that is used for the line-numbers
%  stepnumber=2,                   % the step between two line-numbers. If it's 1, each line 
                                  % will be numbered
  numbersep=5pt,                  % how far the line-numbers are from the code
  backgroundcolor=\color{white},      % choose the background color. You must add \usepackage{color}
  showspaces=false,               % show spaces adding particular underscores
  showstringspaces=false,         % underline spaces within strings
  showtabs=false,                 % show tabs within strings adding particular underscores
  frame=single,                   % adds a frame around the code
  rulecolor=\color{black},        % if not set, the frame-color may be changed on line-breaks within not-black text (e.g. commens (green here))
  tabsize=2,                      % sets default tabsize to 2 spaces
  captionpos=b,                   % sets the caption-position to bottom
  breaklines=true,                % sets automatic line breaking
  breakatwhitespace=false,        % sets if automatic breaks should only happen at whitespace
  title=\lstname,                   % show the filename of files included with \lstinputlisting;
                                  % also try caption instead of title
  keywordstyle=\color{blue},          % keyword style
  commentstyle=\color{dkgreen},       % comment style
  stringstyle=\color{mauve},         % string literal style
  escapeinside={\%*}{*)},            % if you want to add a comment within your code
  morekeywords={*,...}               % if you want to add more keywords to the set
}
\renewcommand\lstlistingname{Листинг}
\renewcommand\lstlistlistingname{Листинги}
\newcommand{\itodo}{\todo[inline]}
\lstloadlanguages{C++}
\DeclareCaptionFont{blue}{\color{blue}} 
%\%captionsetup[lstlisting]{singlelinecheck=false, labelfont={blue}, textfont={blue}}
%\DeclareCaptionFont{white}{\color{white}}
%\DeclareCaptionFormat{listing}{\colorbox[cmyk]{0.43, 0.35, 0.35,0.01}{\parbox{\textwidth}{\hspace{15pt}#1#2#3}}}
%\captionsetup[lstlisting]{
%	format=listing,
%	labelfont=white,
%	textfont=white,
%	singlelinecheck=false,
%	margin=0pt,
%	font={bf,footnotesize}
%}
\numberwithin{equation}{section}
\setcounter{secnumdepth}{3} % глубина нумеруемых разделов
\setcounter{tocdepth}{3} % глубина оглавления

\begin{document}
\begin{center}
	\Large\textbf{{{Вырождение матрицы граничного корректора СХМ в зависимости от угла между нормалью к границе и направлением расчёта}}}
\end{center}

\section{Упругость 3D -- фиксированная внешняя сила}
Искомый вектор решения:
\begin{eqnarray}
	\vec{u}=(v_x,v_y,v_z,\sigma_{xx},\sigma_{xy},\sigma_{xz},\sigma_{yy},\sigma_{yz},\sigma_{zz})^{\small{T}}
\end{eqnarray}

Запись произвольного линейного граничного условия:
\begin{eqnarray}
\label{boundary_condition}
	\mathbf{B} \cdot \vec{u} = \vec{b}.
\end{eqnarray}

Допустим, расчёт производится вдоль оси $x$. Это не уменьшает общности, так как направление нормали к границе $\vec{n}$ выбирается произвольно:
\begin{eqnarray}
	\vec{n} = (n_x, n_y, n_z)^{\small{T}}
\end{eqnarray}

Возьмём три внешних инварианта -- три однонаправленных волны вдоль оси $x$:
\begin{align}
	\mathbf{\Omega} =
	\left( \begin{array}{cccccccccccc}
	-1 & 0  &  0 \\
	 0 & -1 &  0 \\
	 0 &  0 & -1 \\
	\frac{\lambda+2\mu}{c_1} & 0 & 0 \\
	0 & \frac{\mu}{c_2} & 0 \\
	0 & 0 & \frac{\mu}{c_2} \\
	\frac{\lambda}{c_1} & 0 & 0 \\
	0 & 0 & 0 \\
	\frac{\lambda}{c_1} & 0 & 0 \\
	\end{array} \right),
\end{align} 

здесь первый столбец -- p-волна, далее -- две s-волны, $\lambda, \mu$ -- параметры Ламе, $c_1, c_2$ -- скорости продольной и поперечной волн.

Рассматриваем граничное условие фиксированной силы, приложенной к границе:
\begin{eqnarray}
\label{fixed_force_global}
	\mathbf{\sigma} \cdot \vec{n} = \vec{f}.
\end{eqnarray}

Запись $\mathbf{\sigma} \cdot \vec{n}$ даёт выражение для силы, приложенной к полупространству с внешней нормалью $\vec{n}$, в глобальном базисе. Обычно в программе расчёт делается в локальном базисе, связанном с границей, однако это не меняет сути дела (проверено численно в том числе), поэтому далее для упрощения формул будем писать в глобальном базисе.

Для \eqref{fixed_force_global} уравнение граничного условия \eqref{boundary_condition} сводится к $\vec{b} = \vec{f}$ и
\begin{align}
	\mathbf{B} =
	\left( \begin{array}{cccccccccccc}
	 0 & 0 & 0 & n_x & n_y & n_z & 0 & 0 & 0 \\
	 0 & 0 & 0 & 0 & n_x & 0 & n_y & n_z & 0 \\
	 0 & 0 & 0 & 0 & 0 & n_x & 0 & n_y & n_z \\
	\end{array} \right).
\end{align}

После внутреннего расчёта -- учёта волн, пришедших изнутри -- выполняется граничная коррекция. Идея: добавление такой линейной комбинации внешних волн, которая обеспечит выполнение граничного условия.
\begin{eqnarray}
	\mathbf{B} \cdot (\vec{u}_{inner} + \mathbf{\Omega} \cdot \vec{\alpha}) = \vec{b}.
\end{eqnarray}


Как видно, для определения коэффициентов в линейной комбинации необходимо решить систему:
\begin{eqnarray}
	\mathbf{B} \mathbf{\Omega} \cdot \vec{\alpha} = \vec{b} - \mathbf{B} \cdot \vec{u}_{inner}.
\end{eqnarray}

Таким образом, нас интересует обусловленность именно матрицы $\mathbf{B} \mathbf{\Omega}$ размером $D \times D$, где $D$ -- размерность пространства.

После перемножения получаем:
\begin{align}
	\mathbf{B} \mathbf{\Omega} = \frac{1}{c_1 c_2}
	\left( \begin{array}{cccccccccccc}
	 (\lambda + 2\mu) n_x c_2 & \mu n_y c_1 & \mu n_z c_1   \\
	 \lambda n_y c_2          & \mu n_x c_1 & 0             \\
	 \lambda n_z c_2          & 0 & \mu n_x c_1             \\
	\end{array} \right).
\end{align}

\begin{eqnarray}
	\det (\mathbf{B} \mathbf{\Omega}) = \frac{\mu^{2}}{c_2} \cdot n_x \cdot (2 (\lambda + \mu) n^2_x - \lambda)
\end{eqnarray}

Из условия $\det (\mathbf{B} \mathbf{\Omega}) = 0$ получаем:
\begin{eqnarray}
\left[
\begin{gathered} 
	 n_x = 0  \hfill  \\
	 n^2_x = \frac{\lambda}{2(\lambda + \mu)} \equiv \nu < \frac{1}{2}, \hfill  \\
\end{gathered} 
\right.
\end{eqnarray}

где $\nu$ -- коэффициент Пуассона.


\section{Упругость 2D -- фиксированная внешняя сила}
Искомый вектор решения:
\begin{eqnarray}
	\vec{u}=(v_x,v_y,\sigma_{xx},\sigma_{xy},\sigma_{yy})^{\small{T}}
\end{eqnarray}

Два внешних инварианта -- p- и s-волны вдоль оси $x$:
\begin{align}
\label{outer_waves_elastic2d}
	\mathbf{\Omega} =
	\left( \begin{array}{cccccccccccc}
	-1 & 0   \\
	 0 & -1  \\
	\frac{\lambda+2\mu}{c_1} & 0 \\
	0 & \frac{\mu}{c_2} \\
	\frac{\lambda}{c_1} & 0 \\
	\end{array} \right),
\end{align} 

\begin{align}
	\mathbf{B} =
	\left( \begin{array}{cccccccccccc}
	 0 & 0 & n_x & n_y & 0 \\
	 0 & 0 & 0 & n_x & n_y \\
	\end{array} \right).
\end{align}

\begin{align}
	\mathbf{B} \mathbf{\Omega} = \frac{1}{c_1 c_2}
	\left( \begin{array}{cccccccccccc}
	 (\lambda + 2\mu) n_x c_2 & \mu n_y c_1    \\
	 \lambda n_y c_2          & \mu n_x c_1    \\
	\end{array} \right).
\end{align}

\begin{eqnarray}
	\det (\mathbf{B} \mathbf{\Omega}) = \frac{\mu}{c_1 c_2} \cdot (2 (\lambda + \mu) n^2_x - \lambda)
\end{eqnarray}

Из условия $\det (\mathbf{B} \mathbf{\Omega}) = 0$ получаем:
\begin{eqnarray}
	 n^2_x = \nu.
\end{eqnarray}


\section{Упругость -- фиксированная скорость на границе}

В глобальном базисе:
\begin{align}
	\mathbf{B} =
	\left( \begin{array}{cccccccccccc}
	 1 & 0 & 0 & 0 & 0 \\
	 0 & 1 & 0 & 0 & 0 \\
	\end{array} \right).
\end{align}

В локальном базисе $\{\vec{m}, \vec{n}\}$:
\begin{align}
	\mathbf{B} =
	\left( \begin{array}{cccccccccccc}
	 m_x & m_y & 0 & 0 & 0 \\
	 n_x & n_y & 0 & 0 & 0 \\
	\end{array} \right).
\end{align}

Как видно из \eqref{outer_waves_elastic2d}, в случае фиксированной скорости первые $D$ столбцов матрицы $\mathbf{B}$, которые сами являются векторами локального или глобального базиса, умножаются на плюс-минус единичную матрицу, поэтому в данном случае всегда
\begin{eqnarray}
	 \det (\mathbf{B} \mathbf{\Omega}) = \pm 1.
\end{eqnarray}


\section{Акустика}
Искомый вектор решения:
\begin{align}
	\vec{u} =
	\left( \begin{array}{cccccccccccc}
	 \vec{v} \\
	 p \\
	\end{array} \right),
\end{align} 

Независимо от размерности только один внешний инвариант -- p-волна вдоль направления $\vec{l}$:
\begin{align}
	\mathbf{\Omega} =
	\left( \begin{array}{cccccccccccc}
	 \vec{l} \\
	 c \rho \\
	\end{array} \right),
\end{align} 

Матрица для фиксированного давления:
\begin{align}
	\mathbf{B_p} =
	\left( \begin{array}{cccccccccccc}
	 \vec{0}^T & 1 \\
	\end{array} \right), \quad
	\mathbf{B_p} \mathbf{\Omega} = c \rho > 0.
\end{align}

Матрица для фиксированной нормальной скорости:
\begin{align}
	\mathbf{B_v} =
	\left( \begin{array}{cccccccccccc}
	 \vec{n}^T & 0 \\
	\end{array} \right), \quad
	\mathbf{B_v} \mathbf{\Omega} = (\vec{n} \cdot \vec{l}),
\end{align}
что в терминах выше означает $n_x = 0$.


\section{Заключение}
Как видно, наивное представление о том, что матрица граничного корректора вырождается только при перпендикулярности нормали и направления расчёта, не соответствует действительности. Более того, она может быть и не вырожденной при таком условии:


\begin{center}
    \begin{tabular}{ | l | l | l | }
    \hline
    Модель & Фикс. сила & Фикс. скорость \\ \hline
    Упругость 3D & $\left[ \begin{gathered} n_x = 0  \hfill \\ n^2_x = \nu \hfill \\ \end{gathered} \right.$
 & - \\ \hline
    Упругость 2D & $n^2_x = \nu$ & -  \\ \hline
    Акустика & - & $n_x = 0$     \\ \hline
    \end{tabular}
\end{center}



















\end{document}



