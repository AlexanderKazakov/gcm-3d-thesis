\documentclass[twocolumn]{article}
%\usepackage[cp1251]{inputenc}
\usepackage[utf8]{inputenc}
\usepackage[english,russian]{babel}
\usepackage{amsmath,amssymb}
\setlength{\textwidth}{170mm} \setlength{\textheight}{242mm}
\topmargin=-10mm
\oddsidemargin=-4mm

\begin{document}

Для решения волновых задач линейной упругости и акустики в неоднородных областях произвольной геометрии предлагается явный метод, основанный на пространственном расщеплении и последующем решении одномерных уравнений с учётом их гиперболических свойств \cite{alexanderkazak_magomedov_kholodov_1988}.

Эффективный способ расчёта граничных и контактных узлов предложен в \cite{alexanderkazak_chelnokov}. Удовлетворение граничных условий на каждом этапе расщепления по пространственным направлениям сводится к решению СЛАУ на амплитуды волн, условно пришедших извне области.
Однако, как показывает данная работа, этот способ чувствителен к вырождению матрицы системы уравнений, что требует, во-первых, обязательной согласованности граничных и начальных условий в начале каждого этапа расщепления по направлениям, во-вторых, явного учёта вырожденных случаев.

Реализация метода в трёхмерном случае осложняется также тем, что различие между минимальной и максимальной высотами  тетраэдров в расчётной сетке при использовании доступных на сегодня сеточных генераторов составляет несколько порядков. Как альтернатива иерархическим шагам по времени в данной работе применяется расчёт с шагом по времени больше курантовского, что может быть реализовано благодаря сущности сеточно-характеристического метода -- интерполяции по области зависимости решения.

Наконец, поиск ячейки неструктурированной сетки для интерполяции значения на предыдущем временном слое представляет собой неустойчивый алгоритм, что требует особого учёта вырожденных случаев.

На основе предложенного метода реализован программный комплекс, позволяющий проводить моделирование распространения звуковых волн в организме человека в 3D с явным выделением разных типов тканей \cite{alexanderkazak_transcranial_ultrasound}. Работа над методом продолжается.

\begin{thebibliography}{9}

\bibitem{alexanderkazak_magomedov_kholodov_1988} Магомедов К.М., Холодов А.С. Сеточно-характеристические численные методы. — М.: Наука, 1988, 288 с.

\bibitem{alexanderkazak_chelnokov} Челноков Ф.Б., Явное представление сеточно-характеристических схем для уравнений упругости в двумерном и трехмерном пространствах, Матем. моделирование, 18:6, 2006, 96–108.

\bibitem{alexanderkazak_transcranial_ultrasound} K. A. Beklemysheva, G. K. Grigoriev, N. S. Kulberg, A. O. Kazakov, I. B. Petrov, V. Yu. Salamatova, Y. V. Vassilevski, A. V. Vasyukov. Transcranial ultrasound of cerebral vessels in silico: Proof of concept // Russian Journal of Numerical Analysis and Mathematical Modelling 31(5) September 2016

\end{thebibliography}
\end{document}
