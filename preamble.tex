\section*{Введение}
\addcontentsline{toc}{section}{Введение}
В данной работе рассматривается метод численного моделирования низкоскоростного
удара по многослойной инженерной конструкции. Под низкоскоростным ударом
понимается такое столкновение двух тел, при котором не происходит заметных
деформаций.

Многослойные конструкции нашли применение во многих технических областях. Одним
из самых ярких примеров являются композиционные материалы -- искусственно
созданные  материалы, состоящие из нескольких слоёв с чёткой границей раздела
между ними. Композиционные материалы используются в авиации и космонавтике: они
применяются для изготовления обшивок воздушных и космических аппаратов, а также
при создании силовых конструкций, испытывающих серьёзные нагрузки. Композитные
материалы одновременно обладают лёгкостью и заметной прочностью, поэтому всё
чаще они используются для создания брони военной техники и бронежилетов.
Преградой на пути к массовому использованию композиционных материалов является
отсутствие экспериментальных методов проверки характеристик таких материалов.
Так, давно опробованные и использующиеся методы измерения прочностн\'{ы}х
характеристик металлов, не подходят для анализа композитов: композиты по своей
природе анизотропны.

Одной из актуальных прикладных задач, связанных с испытанием многослойных
материалов, является задача о получении волновой картины в образце после
непробивающего удара. В силу анизотропности свойств многослойные материалы после
нагрузок могут заметно терять прочность даже при отсутствии видимых поврежедний.
Это обусловлено появлением микротрещин, которые впоследствии, объединяясь,
превращаются в макротрещины. Так, возникающее после нагрузки расслоение
материала может быть визуально не заметно, хотя делает образец непригодным к
дальнейшему использованию.

Поведение композиционных материалов при динамических нагрузках хорошо
описывается уравнениями механики деформируемого твёрдого тела. Решить эти
уравнения аналитически в большинстве случаев не представляется возможным ввиду
сложности геометрии подобных задач. Поэтому в последнее время для испытания
композиционных материалов используют следующий подход:
\begin{itemize}
\item проводят эксперимент по нагружению опытного образца;
\item моделируют этот эксперимент численно;
\item проводят валидацию результатов моделирования значениями, полученными в ходе эксперимента;
\item при прохождении проверки используют результаты численного моделирования
для получения картины динамической деформации образца.
\end{itemize}

Динамическое воздействие вызывает распространение упругих волн в образце. В случае 
композитного материала множественные переотражения волн от внутренних контактных 
границ между слоями создают сложную волновую картину. Интерференция прямых, отражённых 
и преломлённых волн формирует итоговые области максимальных нагрузок в образце. В связи
с этим для моделирования необходимо использовать численный метод решения системы 
уравнений механики деформируемого твёрдого тела, позволяющий получить полную волновую 
картину с высоким временным и пространственным разрешением с учётом влияния контактных 
границ.

Указанными свойствами обладает сеточно-характеристический численный метод. Поэтому 
в данной работе реализован программный комплекс для высокопроизводительных вычислений 
в области механики деформируемого твёрдого тела с использованием сеточно-характеристического 
метода.
