\section*{Введение}
\addcontentsline{toc}{section}{Введение}
В данной работе рассматривается численное моделирование волновых деформационных процессов в твёрдых телах с пластической реологией. Рассчёты реализованы для одномерной модели -- идеальная пластика, движение рассчётной сетки со вторым порядком по времени, и для трёхмерной модели -- жёсткопластическая среда с критерием текучести Мизеса, движение сетки на этапе написания диплома находится в разработке. 

Для описание физических процессов используются уравнения на скорости и напряжения в частицах материала из механики деформируемого твёрдого тела. В качестве модели пластической реологии выбрана модель пластического течения без упрочнения с критерием текучести Мизеса. Она особенно подходит для описания металлов, но также может в нулевом приближении быть применена к пластикам и другим материалам.

Рассчёт имеющейся системы уравнений проводится расщеплением на физические процессы упругой деформации и пластического течения. Линейно-упругая часть решается сеточно-характеристическим методом с расщеплением по трём пространственным направлениям. Пластическая часть сводится к нормирующему тензор напряжений корректору. 

Интерференция деформационных волн может привести к высоким значениям напряжения, при которых происходит разрушение материала. Такие явления не могут быть замечены в статическом рассмотрении. 

Сеточно-характеристический метод построен на основе гиперболических свойств изучаемых уравнений и благодаря этому позволяет воспроизводить точную динамическую волновую картину на малых временах, что совершенно необходимо при изучении ударных воздействий на материалы. 


\begin{itemize}
\item а
\item б
\item в
\item г
\end{itemize}
