\section*{Введение}
\addcontentsline{toc}{section}{Введение}
В данной работе рассматриваются существующие подходы к описанию деформируемого твёрдого тела, методы численного решения получаемых уравнений и реализация на их основе компьютерного моделирования волновых деформационных процессов в твёрдых телах с пластической реологией с помощью расщепления по координатам и по физическим процессам и сеточно-характеристического метода. 

Расчёты реализованы для одномерной модели -- линейная упругость и упругопластика, движение тела и рассчётной сетки со вторым порядком по времени, произвольные граничные условия. Для случая трёх измерений на основе существующего программного комплекса для расчёта линейной упругости реализован метод моделирования среды с идеально-пластической реологией. Движение сетки на этапе написания диплома находится в разработке. 

Для описание физических процессов используются уравнения на скорости и напряжения в частицах материала из механики деформируемого твёрдого тела. В качестве модели пластической реологии выбрана модель пластического течения без упрочнения с критерием текучести Мизеса. Она особенно подходит для описания металлов, но также может в нулевом приближении быть применена к пластикам и другим материалам.

Расчёт имеющейся системы уравнений проводится расщеплением на физические процессы упругой деформации и пластического течения. Линейно-упругая часть решается сеточно-характеристическим методом с расщеплением по трём пространственным направлениям. Пластическая часть сводится к нормирующему тензор напряжений корректору. 

Интерференция деформационных волн может привести к высоким значениям напряжения, при которых происходит разрушение материала. Такие явления не могут быть замечены в статическом, и должны рассчитываться в динамическом (или волновом) рассмотрении.

Сеточно-характеристический метод построен на основе гиперболических свойств изучаемых уравнений и благодаря этому позволяет воспроизводить точную волновую картину на малых временах, что совершенно необходимо при изучении ударных воздействий на материалы.
