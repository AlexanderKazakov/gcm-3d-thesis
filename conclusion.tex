\section{Заключение}
Результаты, полученные в результате выполнения дипломного проекта:
\begin{itemize}
\item Рассмотрены различные подходы к описанию движения сплошной среды в сочетании с применяемыми к ним численными методами
\item Предложен метод моделирования сплошной среды в 1D с учётом движения и деформации тела, в том числе схема второго порядка точности по координате с учётом изменения характеристик среды в пространстве, точная со вторым порядком по времени схема расщепления рассчётного шага на движение рассчётной сетки и перенос возмущения
\item На основе данного метода реализован программный комплекс, моделирующий деформируемое твёрдое тело в случае одного пространственного измерения с поддержкой упругой и упругопластической реологии
\item Проведено сравнение результатов работы данного программного комплекса с результатами полуаналитического рассчёта на основе принципиально другого подхода к описанию сплошной среды и итерационного  метода. Получена сходимость результатов со вторым порядком точности по времени и координате
\item Предложен способ моделирования идеально-пластической реологии в 3D на основе правила корректировки Уилкинса
\item Данный способ реализован на основе существующего программного комплекса, поддерживающего линейную упругость в 3D
\item Проведено сравнение результатов. Получено удовольствие
\end{itemize}
