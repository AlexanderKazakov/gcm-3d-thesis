\section{Заключение}
Основные результаты данной работы:
\begin{itemize}
\item Реализован и проверифицирован расширяемый программный комплекс для численного моделирования динамических задач механики деформируемого твёрдого тела
\item Программный комплекс применён к моделированию волновых процессов при ультразвуковом неразрушающем контроле изделий из изотропных и анизотопных композиционных материалов
\item Проведены расчёты постановок неразрушающего контроля анизотропных композиционных материалов с реальными параметрами образцов
\item Смоделированы показания ультразвукового детектора для неповреждённого образца и образца с дефектом
\item Предложен метод интерполяции второго порядка на расчётных сетках на триангуляциях, не требующий введения вспомогательных узлов на рёбрах сетки
\item Предложен алгоритм поиска точки пересечения характеристики с временными слоями, учитывающий невыпуклость области интегрирования и пересечение с границами области интегрирования
\end{itemize}
