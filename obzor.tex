\section{Обзор существующих работ по данной тематике}

Комбинация математической модели линейно-упругого тела и сеточно-характеристического численного метода для расчёта волновых задач используется уже давно и хорошо себя зарекомендовала.

Сеточно\hyp{}характеристический метод численного решения уравнений в частных производных гиперболического типа был впервые предложен в \cite{magomedov_kholodov_1969}. В последствии его авторами была выпущена подробная монография по различным аспектам его применения и реализации \cite{magomedov_kholodov_1988}.

Одно из первых применений метода к расчёту волновых задач механики деформируемого твёрдого тела было осуществлено в \cite{petrov_kholodov} на структурированных расчётных сетках. Возможность применения метода с высоким порядком аппроксимации на неструктурированных расчётных сетках была продемонстрирована в работе \cite{chelnokov_agapov}. Значительные улучшения метода, связанные со спектральным разложением матриц системы уравнений и расчётом граничных и контактных узлов, были предложены в \cite{chelnokov}. В работе \cite{favorskaya_anysotropy} был предложен способ диагонализации матриц системы уравнений для произвольного случая анизотропии материала. Реализация метода на  структурированных иерархических тетраэдральных сетках с кратным шагом по времени продемонстрирована в работе \cite{favorskaya}.

В монографии \cite{kulikovskiy} рассмотрены различные математические вопросы сеточно\hyp{}характеристического метода и других родственных ему методов численного решения динамических волновых задач.

Таким образом, данная область знаний уже является хорошо изученной, однако полная реализация всех методов в трёхмерном пространстве с различной реологией материалов, сложной геометрией областей и другими практическими аспектами, важными в прикладных задачах, всё ещё оставляет много вопросов.