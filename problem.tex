\section{Постановка задачи}
В данной работе рассматривается метод численного моделирования низкоскоростного удара по многослойной инженерной конструкции. Под низкоскоростным ударом понимается такое столкновение двух тел, при котором не происходит заметных деформаций. 

Многослойные конструкции нашли применение во многих технических областях. Одним из самых ярких примеров являются композиционные материалы -- искусственно созданные  материалы, состоящие из нескольких слоёв с чёткой границей раздела между ними. Композиционные материалы используются в авиации и космонавтике: они применяются для изготовления обшивок воздушных и космических аппаратов, а также при создании силовых конструкций, испытывающих серьёзные нагрузки. Композитные материалы одновременно обладают лёгкостью и заметной прочностью, поэтому всё чаще они используются для создания брони военной техники и бронежилетов. Преградой на пути к массовому использованию композиционных материалов является отсутствие экспериментальных методов проверки характеристик таких материалов. Так, давно опробованные и использующиеся методы измерения прочностн\'{ы}х характеристик металлов, не подходят для анализа композитов: композиты по своей природе анизотропны. Поэтому решить задачу о получении волновой картины в многослойном материале после низкоскоростного удара на данный момент возможно только при помощи численного моделирования.

В основном в области численного моделирования пространственно-волновых процессов в многослойных средах используются обычные конечно-разностные аппроксимации производных по времени и координате с заданным порядком точности. Информация о том, как ведут себя такие схемы вблизи многочисленных контактных границ или на динамических разрывах, монотонности схем весьма ограничена, либо отсутствует.

Одними из наиболее широко используемых вычислительных методов для решения систем уравнений гиперболического типа механики сплошных сред являются сеточно-характеристические методы, подробное описание и обзор которых можно найти в \cite{magomedov}. \itodo{Пруф?}Главная особенность, присущая методу характеристик, -- нерегулярность разностной сетки -- оказалась серьёзным препятствием для обобщения этого метода на многомерный случай. Существенное продвижение здесь было получено в работах \cite{chushkin}, основанных на сочетании метода характеристик и конечно-разностных подходов.

В работе \cite{petrov_chelnokov} рассматриваются особенности протекания процессов деформирования в многослойных преградах конечной толщины, вызванных ударом абсолютно твёрдого сферического тела. Для моделирования поведения материала преграды использовались реологические модели линейно-упругой среды (закон Гука), упругопластической (модель \itodo{Кто все эти люди?}Прандтля-Рейса с условиями пластичности Мизеса и Мизеса-Шлейхера, модель Маквелла), упруговязкопластической сред. Характерной особенностью работы является использование модели разрушения (модель Майнчена-Сака), а также использование различных подходов к перестроению сеток. Для численного решения использовался сеточно-характеристический метод, гибридная и гибридизованная разностные схемы. Минусами является использование структурированной (прямоугольной) сетки и только двух пространственных переменных.

В работе \cite{matyushev_petrov} проводилось численное исследование волновых и деформационных процессов в многослойных  средах. Как и в предыдущей работе, использовался численно-характеристический метод, а также \itodo{А какая разница?}гибридная и гибридизированная схемы. Особенностью является использование неструктурированных (треугольных) сеток, а также применение сеточно-характеристического шаблона на неструктурированных сетках (этот подход весьма необычен, так как использование шаблона предполагает наличие структурированной сетки). Моделировался удар деформируемым сферическим ударником по многослойной (5 – 20 слоев) преграде. Данная модель описывала пуленепробиваемый жилет и человеческое тело, защищённое им. Целью было найти оптимальные параметры среды для максимального поглощения воздействия ударника. К недостаткам работы можно отнести моделирование лишь по двум пространственным координатам.

В \cite{petrov_tormasov_holodov} рассматрены одномерные и двумерные нестационарные задачи о действии ударных и других нестационарных нагрузок на деформируемые твёрдые среды многослойной структуры, описаны возникающие волновые и откольные эффекты. Для описания поведения среды использованы модели линейно упругого и упругоидеальнопластического тела. На поверхностях раздела слоев рассматрены условия трёх типов: полного слипания, свободного скольжения, отслоения. В работе изучалось в основном влияние многослойных структур на амплитуду проходящей волны. На основании одномерных расчётов в слоистых средах был сделан вывод, что слоистые конструкции можно использовать для уменьшения амплитуды волн и для увеличения сжимающих напряжений вблизи лицевой поверхности, например, для увеличения силы сопротивления.

В \cite{golubev_kvasov_petrov} исследуются задачи распространения упругих волн, возникающих в результате землетрясения, в различных геологических средах: однородной, многослойной, градиентной, с трещиноватым пластом и карстовым образованием. Также проводится анализ воздействия упругих волн на поверхностные промышленные сооружения: здания и плотины. Проводится сравнение воздействия упругих волн на дневную поверхность для различных геологических сред. Качественно рассматривается влияние упругих волн на прочность поверхностных сооружений. В работе используется сеточно-характеристический метод на треугольных сетках, контактные границы выделяются явно.

В \cite{agapov_belocerkovsky_petrov} сформулирована двумерная математическая модель механической реакции головы человека на ударные воздействия, описывающая пространственное распределение механических нагрузок на мозг (который в принципе является многослойной средой). Приведены некоторые результаты ее численного исследования с применением сеточно-характеристических методов на структурированных (прямоугольных) и неструктурированных (треугольных) сетках.

Работа \cite{petrov} посвящена численному исследованию волновых и откольных явлений, возникающих при импульсном нагружении двух- и четырехслойных упругопластических цилиндрических оболочек конечной толщины, подкрепленных с тыльной стороны ребрами жесткости. Используется динамическая система двумерных уравнений механики деформируемого тела и упруго идеально пластическая модель Прандтля-Рейсса. В работе показана возможность не только получать разрушенные зоны, но и отдельные трещины, зоны концентраций напряжений вблизи трещины и края откольной зоны, которая переходит в трещину, являющуюся самостоятельным источником нестационарных возмущений.

Автору неизвесты работы на схожую тематику, использующие сеточно-характеристический метод на неструктурированных сетках при трех пространственных переменных с явным выделением контактных границ.

