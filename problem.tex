\section{Обзор существующих работ по данной тематике}

Для динамических задач механики деформируемого твердого тела необходимо использование численных методов, позволяющих получить полную волновую картину с высоким временным и пространственным разрешением с учётом влияния контактных границ. Так как определяющая система уравнений в частных производных относится к гиперболическому типу, одними из наиболее используемых вычислительных методов для её решения являются сеточно-характеристические методы, подробное описание и обзор которых можно найти в \cite{magomedov}. Существенное продвижение было получено в работах \cite{chushkin}, основанных на сочетании метода характеристик и конечно-разностных подходов.

В работе \cite{petrov_chelnokov} рассматриваются особенности протекания процессов деформирования в многослойных преградах конечной толщины, вызванных ударом абсолютно твёрдого сферического тела. Для моделирования поведения материала преграды использовались реологические модели линейно-упругой среды (закон Гука), упругопластической (модель Прандтля-Рейса с условиями пластичности Мизеса и Мизеса-Шлейхера, модель Максвелла), упруговязкопластической сред. Характерной особенностью работы является использование модели разрушения (модель Майнчена-Сака), а также использование различных подходов к перестроению сеток. Для численного решения использовался сеточно-характеристический метод, гибридная и гибридизованная разностные схемы. Минусами является использование структурированной (прямоугольной) сетки и только двух пространственных переменных.

В работе \cite{matyushev_petrov} проводилось численное исследование волновых и деформационных процессов в многослойных  средах. Как и в предыдущей работе, использовался численно-характеристический метод, а также гибридная и гибридизированная схемы. Особенностью является использование неструктурированных (треугольных) сеток, а также применение сеточно-характеристического шаблона на неструктурированных сетках (этот подход нетипичен для данного метода, так как использование шаблона предполагает наличие структурированной сетки). Моделировался удар деформируемым сферическим ударником по многослойной (5 – 20 слоев) преграде. Данная модель описывала бронежилет и человеческое тело, защищённое им. Целью было найти оптимальные параметры среды для максимального поглощения воздействия ударника. К недостаткам работы можно отнести моделирование лишь по двум пространственным координатам.

В \cite{petrov_tormasov_holodov} рассмотрены одномерные и двумерные нестационарные задачи о действии ударных нагрузок на деформируемые твёрдые среды многослойной структуры, описаны возникающие волновые эффекты, приводящие к появлению откола. Для описания поведения среды использованы модели линейно упругого и упругоидеальнопластического тела. На поверхностях раздела слоев рассматрены условия трёх типов: полного слипания, свободного скольжения, отслоения. В работе изучалось в основном влияние многослойных структур на амплитуду проходящей волны. На основании одномерных расчётов в слоистых средах был сделан вывод, что слоистые конструкции можно использовать для уменьшения амплитуды волн и для увеличения сжимающих напряжений вблизи лицевой поверхности, например, для увеличения силы сопротивления.

В \cite{golubev_kvasov_petrov} исследуются задачи распространения упругих волн, возникающих в результате землетрясения, в различных геологических средах: однородной, многослойной, градиентной, с трещиноватым пластом и карстовым образованием. Также проводится анализ воздействия упругих волн на поверхностные промышленные сооружения: здания и плотины. Проводится сравнение воздействия упругих волн на дневную поверхность для различных геологических сред. Качественно рассматривается влияние упругих волн на прочность поверхностных сооружений. В работе используется сеточно"=характеристический метод на треугольных сетках, контактные границы выделяются явно.

В \cite{agapov_belocerkovsky_petrov} сформулирована двумерная математическая модель механической реакции головы человека на ударные воздействия, описывающая пространственное распределение механических нагрузок на мозг (система череп-мозг с точки зрения механики также является многослойной средой). Приведены некоторые результаты ее численного исследования с применением сеточно-характеристических методов на структурированных (прямоугольных) и неструктурированных (треугольных) сетках.

Работа \cite{petrov} посвящена численному исследованию волновых процессов и явления откола, возникающих при импульсном нагружении двух- и четырехслойных упругопластических цилиндрических оболочек конечной толщины, подкрепленных с тыльной стороны ребрами жесткости. Используется динамическая система двумерных уравнений механики деформируемого тела и упруго идеально пластическая модель Прандтля-Рейсса. В работе показана возможность не только получать разрушенные зоны, но и отдельные трещины, зоны концентраций напряжений вблизи трещины и края откольной зоны, которая переходит в трещину, являющуюся самостоятельным источником нестационарных возмущений.

В работе \cite{vasyukov} рассматривается моделирование волновых процессов МДТТ в трёхмерном случае сеточно-характеристическим методом на неструктурированных тетраэдрических сетках. На основе описываемого метода реализован программный комплекс "gcm-3d" (https://github.com/avasyukov/gcm-3d), моделирующий линейно-упругие возмущения. Этот комплекс был взят за основу данного дипломного проекта для реализации модели пластического течения.
